\documentclass[10pt,a4paper,draft]{article}
\usepackage[utf8]{inputenc}
\usepackage[english]{babel}
\usepackage{amsmath}
\usepackage{amsfonts}
\usepackage{amssymb}
\usepackage{graphicx}
\author{autogenerated by MDSip-test tools}
\title{MDSip throughput tests report}


\begin{document}

\maketitle

\section{Software setup}

This testing software has been developed upon the MDSplus framework C++ object libraries. It aims to provide a flexible layer to perform different timing tests on the remote segment writing of MDSplus mdsip server.
The tests are focused on finding an optimal recipe of the segment size and the connection charatcteristics tuned with a internet long-distance endpoint.
The overall testing software is organized in a test-oriented library and testing recipes binaries. The library provides some new utility classes that were implemented to obtain an easy to use testing environment. 
Considering the tiny size of the project we defined those classes without a namespace; the structural organization is done simply adding a suffix “Utility” to the source files. 
Beside the utilities functions the test chain is composed by three entity types: the Content, the Channel, and the TestConnection. The content represents the content of the segment to be sent, it can be a function generator (sine or white/gaussian noise for now) or a parse tree reader. The channel is responsable to actually send the segment to the endpoint, two channel flavor can be choosed using Distribute Client (DC) or Thin Client (TC) connections. Finally the TestConnection is a class that collects all different contents and channels and starts the connection itself. Two versions of TestConnection are implemented to manage parallel channels, the TestConnectionMT uses multiple threads (one per channels) and the TestConnectionMP uses different forked process.
Some time and speed histograms are instanced by the TestConnection one per added channels, they accounts the timings of each put segment operation performed by the channel. This feature was introduced to produce a speed inforamtion that remains disjoint from all the tree funcionalities operations performed. In this sense all the results that will follows are representing the equivalent speed of KB per seconds of tansmitted Float32 data. Obviously this data we expect to be lower bounded by the actual network capacity and a comparison test will be done using iperf results.

\section{ Optimal segment size vs protocol }

Tests has been performed using different algorithms TCP and UDT. The testing software talks to a 10Gbps connected machine located at nifs (\emph{test10g.nifs.ac.jp}). This target endpoint and has been opened for connections coming from \emph{ra22.igi.cnr.it} (Red Hat Enterprise Linux Server release 5.1) server at RFX. The nifs backend referent Nakanishi, Hideya <nakanisi@nifs.ac.jp> declared the following firewall rules:

\begin{table}
{ \footnotesize
-A INPUT -m state --state ESTABLISHED,RELATED -j ACCEPT \\
-A INPUT -m state --state NEW -m tcp -p tcp --dport 22 -s 150.178.3.150 -j ACCEPT \\
-A INPUT -m state --state NEW -m tcp -p tcp --dport 8000 -s 150.178.3.150 -j ACCEPT \\
-A INPUT -m state --state NEW -m udp -p udp --dport 8000 -s 150.178.3.150 -j ACCEPT
}
\end{table}


For what concerns the TCP connection, as it can be seen in Figure 1, a single channel shows an increasing speed vs segment size reaching a plateau very soon for about 128KB window size.



\end{document}